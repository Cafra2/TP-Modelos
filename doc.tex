\documentclass[12pt]{article}
\usepackage{graphicx} % Required for inserting images
\usepackage{amsmath, amssymb, amsthm}
\usepackage[spanish]{babel} % Para reglas de español
\usepackage{enumitem} % Para personalizar listas

\title{{TP-Modelos}}
\author{Gabriel Guimpelevich\and Agustín Farace}
\date{Junio 2025}

\begin{document}

\maketitle

\section{Introducción}
Este es un trabajo práctico en el que se busca estudiar la efectividad de varios métodos de generación de números pseudoaleatorios al compararlos con análisis estadístico y aplicarlos a un problema concreto de simulación.
\\\\
    \subsection{Generadores usados}
    Los generadores elegidos son:
    \begin{itemize}
        \item Generador Congruencial Lineal (LCG) con parámetros:
        \begin{itemize}[label=$\ast$]
            \item $a=16807$ 
            \item $c=0$ 
            \item $m=2^{31}-1$
        \end{itemize}
    
        \item XORShift
        \item Mersenne Twister MT19937
    \end{itemize}
    
\pagebreak
    \subsection{Problema a simular}
    Se desea estimar, mediante el método de Monte Carlo, el valor de la siguiente integral múltiple:
    \[
    \int_{[0,1]^5} \left( \sum_{i=1}^5 x_i \right)^2 \, dx_1...dx_5
    \]
    Esta integral representa el valor esperado de la suma al cuadrado de cinco variables independientes uniformemente distribuidas en [0,1].\\
    Para referencia, el valor teórico de la integral es $\approx \frac{20}{3}$.
\end{document}
